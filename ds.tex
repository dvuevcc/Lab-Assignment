
\documentclass{rmf-d}
\usepackage{nopageno,rmfbib,multicol,times,epsf,amsmath,amssymb,cite}
\usepackage[T1]{fontenc} %Especial para español (for spanish)
%\usepackage[spanish, mexico]{babel}
\usepackage[]{caption2}
\usepackage{graphicx}
\usepackage{blindtext}
\usepackage{color}
\usepackage{hyperref}
\usepackage{minted}

\def\rmfcornisa{© DEC 2019 | IRE Journals | Volume 3 Issue 6 | ISSN: 2456-8880}

\newcommand{\ssc}{\scriptscriptstyle}
%
\def\rmfcintilla{{\it Rev.\ Mex.\ Fis.\/} {\bf ??} (*?*) (????) ???--???}
\clearpage \rmfcaptionstyle \pagestyle{myheadings}
\setcounter{page}{1}

\begin{document}
\markboth{ RMF Editorial Team    }{ A \LaTeX template for the RMF, RMF-E, SRMF }

\title{Data Structures and Its Limitations
\vspace{-6pt}}
\author{  VISHAL V. MEHTRE 1
, UDBHAV SINGH 2       }
\address{1 Assistant Professor, Department of Electrical Engineering, Bharati Vidyapeeth Deemed University,
College of Engineering, Pune, India}
\address{2 Assistant Professor, Department of Electrical Engineering, Bharati Vidyapeeth Deemed University,
College of Engineering, Pune, India
}

\author{ }
\address{ }
\author{ }
\address{ }
\author{ }
\address{ }
\author{ }
\address{ }
\author{ }
\address{ }
\maketitle

\begin{abstract}
Abstract
\end{abstract}

\vspace{-8pt}
\begin{multicols}{2}

In this growing age of technology, computers
play one of the most important roles. Be it medical
surgeries to handling of huge power grids and power
generation stations, all are controlled by computers. The
most basic requirement of a computer to carry out its task
is data. The data that is provided is initially in the raw
form, but in order for it to work with other machinery that
is required to carry out the task and produce an output the
data needs to be structured. This structuring of data is
done by some simple commands and algorithms termed as
data structure. [1] This review paper aims at introducing
different types of data structures as well as addressing the
shortcomings of certain data structure. I will also try to
point out certain improvements that can be carried out in
certain data structures.

\section{Introduction}

DATA STRUCTURES:
Data structure is a method of structuring of data for
easy usage and retrieval. The basic aim of data
structure is to collect data values at a single place,
establish relationship between the data and the
different functions and operations that can be carried
out on the collected data. 
[2]The various types of data
structures are mentioned below along with a brief
introduction to each:-
LINEAR DATA STRUCTURE:
1. Array: - An array is the most basic data structure.
It stores values adjacent to each other ie.
Contiguous memory locations. The address of the
data value next to the selected Data value can be
easily retrieved by incrementing the address of the
selected Data value by one. The data types of all
elements in array are same.
Real life example: - Array can be understood similar
to a staircase where the first Stair is the base value of

the array and each and every subsequent data value is
Similar to following stairs.
Fig1. Basic structure of array
2. Stack: - Stack is a type of linear data structure ie.
All the operations that are to be performed on the
data values are done In a particular order. The order
that may be Followed LIFO (Last in First Out) or
FILO (First in Last Out). [5] In stack the data values
are added on top of each other and so there is a top
value that keeps a track on the number of elements
present in the stack. Real life example: - Books
vertically placed on top of each other.
\begin{figure}
\centering
\includegraphics[width=0.3\textwidth]{NSTU.jpg}
\caption{\label{fig:NSTU}This picture was uploaded via the file-tree menu.}
\end{figure}


Mathematical expressions are intruduced in standard way as in   \mintinline{latex}{$x^2$} that appears as $x^2$. Unnumbered expressions like 
$$a x^2 + b x + c = 0 $$

Numbered equations 
\begin{equation}
    E=mc^2 \label{eq:Einstein}
\end{equation}
\section{Figures, Tables}
 \begin{figure}[H]
 \includegraphics[width=\linewidth]
   {NSTU.jpg}
 \caption{NSTU view from LAL BUS}
 \label{fig:NSTU.jpg}
\end{figure}
Tables are introduced as in the following example

\begin{table}[H]
 \centering
 \caption{True-false Table}
 
 \begin{tabular}{cc|c}
  $p$&$q$&$p\land q$\\
  \hline
  0&0&0\\
  0&1&0\\
  1&0&0\\
  1&1&1
 \end{tabular}\label{table1}
\end{table}
As usual, tables can be crossreferenced in the text~\ref{table1}.



\end{multicols}
\end{document}
